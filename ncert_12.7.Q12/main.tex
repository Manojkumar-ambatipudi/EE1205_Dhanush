% \iffalse
\let\negmedspace\undefined
\let\negthickspace\undefined
\documentclass[journal,12pt,twocolumn]{IEEEtran}
\usepackage{float}
\usepackage{circuitikz}
\usepackage{cite}
\usepackage{amsmath,amssymb,amsfonts,amsthm}
\usepackage{algorithmic}
\usepackage{graphicx}
\usepackage{textcomp}
\usepackage{xcolor}
\usepackage{txfonts}
\usepackage{listings}
\usepackage{enumitem}
\usepackage{mathtools}
\usepackage{gensymb}
\usepackage{comment}
\usepackage[breaklinks=true]{hyperref}
\usepackage{tkz-euclide} 
\usepackage{listings}
\usepackage{gvv}                                        
\def\inputGnumericTable{}                                 
\usepackage[latin1]{inputenc}                                
\usepackage{color}                                            
\usepackage{array}                                            
\usepackage{longtable}                                       
\usepackage{calc}                                             
\usepackage{multirow}                                         
\usepackage{hhline}                                           
\usepackage{ifthen}                                           
\usepackage{lscape}
\newtheorem{theorem}{Theorem}[section]
\newtheorem{problem}{Problem}
\newtheorem{proposition}{Proposition}[section]
\newtheorem{lemma}{Lemma}[section]
\newtheorem{corollary}[theorem]{Corollary}
\newtheorem{example}{Example}[section]
\newtheorem{definition}[problem]{Definition}
\newcommand{\BEQA}{\begin{eqnarray}}
\newcommand{\EEQA}{\end{eqnarray}}
\newcommand{\define}{\stackrel{\triangle}{=}}
\theoremstyle{remark}
\newtheorem{rem}{Remark}
\begin{document}

\bibliographystyle{IEEEtran}
\vspace{3cm}
\title{NCERT 12.7 12Q}
\author{EE23BTECH11015 - DHANUSH V NAYAK$^{*}$% <-this % stops a space
}
\maketitle
\newpage
\bigskip
\renewcommand{\thefigure}{\arabic{figure}}
\renewcommand{\thetable}{\theenumi}
\bibliographystyle{IEEEtran}

\textbf{Question:} An LC circuit contains a $50 \mu H$ inductor and a $50 \mu F$ capacitor with an initial charge of $10 mC$. The resistance of the circuit is negligible. Let the instant the circuit is closed by $t = 0$.

\textbf{a)} What is the total energy stored initially? Is it conserved during LC oscillations?

\textbf{b)} What is the natural frequency of the circuit?

\textbf{c)} At what time is the energy stored \textbf{(i)} completely electrical (i.e., stored in the capacitor)? \textbf{(ii)} completely magnetic (i.e., stored in the inductor)?

\textbf{d)} At what times is the total energy shared equally between the inductor and the capacitor?

\textbf{e)} If a resistor is inserted in the circuit, how much energy is eventually dissipated as heat? \\
\hfill(NCERT 12.7 12Q)\\
\solution
\begin{table}[H]
\centering
\renewcommand\thetable{1}
\setlength{\extrarowheight}{9pt}
\resizebox{0.51\textwidth}{!}{
\begin{tabular}{|c|c|c|}
\hline
\textbf{Parameter} & \textbf{Description} & \textbf{Value} \\ \hline
$L$ &Inductance&$50 \mu H$  \\ \hline
$C$ &Capacitance&$50 \mu F$  \\ \hline
$E\brak{0}$ &Initial Energy of Capacitor&$?$  \\ \hline
$q\brak{0}$ &Initial Charge on Capacitor&$10 mC$  \\ \hline
$f$ &Natural Frequency&$\frac{1}{2\pi\sqrt{LC}}$  \\ \hline
\end{tabular}}
\caption{Parameter Table}
\label{tab:ncert-12.7.12}
\end{table}


\begin{figure}[H]
    \centering
    \resizebox{0.2\textwidth}{!}{\begin{circuitikz}
    \draw (0,0)
    to[L, l=$L$] (2,0)
    to[C, l=$C$] (2,-2)
    -- (0,-2)
    --(0,0);
\end{circuitikz}}
    \caption{LC Circuit}
    \label{fig:ncert_12.7.12_cktdiag}
\end{figure}

\begin{enumerate}[label=\textbf{(\alph*)}]
    \item Initial energy stored :
    \begin{align}
        E\brak{0}&= \frac{{q\brak{0}}^2}{2C}\\
            &=  1 J 
    \end{align}
\begin{figure}[H]
    \includegraphics[width=1\columnwidth]{figs/Plot_energy.png}
    \caption{Energy is Conserved During Oscillations total energy being limited to the initial energy}
    \label{fig:energy_plots}
\end{figure}
 \item Natural frequency:\\
 From \tabref{tab:ncert-12.7.12}:
   \begin{align}
         f&=\frac{1}{2\pi\sqrt{LC}}\\
          &=3183.10 Hz
   \end{align}
The Laplace domain circuit :
\begin{figure}[H]
    \centering
    \resizebox{0.2\textwidth}{!}{\begin{circuitikz}
        \draw (0,0)
        to[american inductor, l=$sL$] (2,0)
        to[capacitor, l=$\frac{1}{sC}$] (2,-2)
        -- (0,-2)
        --(0,0);
    \end{circuitikz}}
    \caption{LC Circuit in lapalace domain}
    \label{fig:ncert_12.7.12_cktdiag_lap}
\end{figure}
The impedance of circuit :
\begin{align}
    Z &= sL + \frac{1}{sC}
\end{align}
s can be expressed in resonant frequency as :
\begin{align}
    s &= 2\pi f\\
    Z &= \brak{2\pi fL - \frac{1}{2\pi fC}}j\\
    \abs{Z} &= 2\pi fL - \frac{1}{2\pi fC}
\end{align}
\begin{figure}[H]
    \includegraphics[width=0.8\columnwidth]{figs/impedance_plot.png}
    \caption{The frequency at which impedance is minimum is natural frequency at $f=3183.10 Hz$}
    \label{fig:Z vs f}
\end{figure}
\item At any given time the charge on the capacitor is given by:
   \begin{align}
       q\brak{t} &= q\brak{0}\cos{2\pi ft},T=\frac{1}{f}\\
       i\brak{t} &= \frac{dq}{dt}\\
                 &= q\brak{0}2\pi f\sin\brak{2\pi ft + \pi}
   \end{align} 
The energy stored is completely electrical when $i\brak{t}=0$. At times $t = 0, \frac{T}{2}, T, \frac{3T}{2}, \ldots$\\

The energy stored is completely magnetic when $q\brak{t}=0$
At times $t =\frac{T}{4},\frac{3T}{4},\frac{5T}{4} \ldots$ .
\begin{figure}[H]
    \includegraphics[width=1\columnwidth]{figs/current_plot.png}
    \caption{Energy is completely electrical at $t = 0, \frac{T}{2}, T, \frac{3T}{2}, \ldots$ }
    \label{fig:current_plot}
\end{figure}
\begin{figure}[H]
    \includegraphics[width=1\columnwidth]{figs/charge_plot.png}
    \caption{Energy is completely magnetic at  $t =\frac{T}{4},\frac{3T}{4},\frac{5T}{4} \ldots$   }
    \label{fig:charge_plot}
\end{figure}
\item When energy is equally shared then the capacitor has half of the maximum energy.
 \begin{align}
        \frac{\brak{q(t)}^2}{2C} &= \frac{1}{2}\\
        {q\brak{0}}^2\brak{\cos{2\pi ft}}^2 &= \frac{1}{2}\\
        \cos{2\pi ft}     &= \frac{1}{\sqrt{2}}\\
        t                 &= \frac{(2n+1)T}{8}
\end{align}
Hence, the total energy is equally shared between the inductor and capacitor at the time,$t =\frac{T}{8},\frac{3T}{8},\frac{5T}{8} \ldots$ \\
\begin{figure}[H]
    \includegraphics[width=1\columnwidth]{figs/Plot_equal_energy.png}
    \caption{At the intersection points with 0.5 J horizontal line both capacitor and inductor have equal energy.}
    \label{fig:equal_energy_plot}
\end{figure}
\item Once the resistor is added to the LC circuit, it starts dissipating energy in the form of heat.
\begin{figure}[H] 
    \centering
    \begin{circuitikz}
        \draw (0,0)
        to[L, l=$L$] (2,0)
        to[R, l=$R$] (2,-2)
        to[C, l=$C$] (0,-2)
        -- (0,0);
    \end{circuitikz}
\end{figure}
Eventually, all the initial energy is dissipated in the form of heat.
\end{enumerate}
\begin{figure}[H]
    \includegraphics[width=1\columnwidth]{figs/Charge across capacitor during Damping.png}
    \caption{Charge Across Capacitor decays as time proceeds}
    \label{fig:charge_during_damping}
\end{figure}

\begin{figure}[H]
    \includegraphics[width=1\columnwidth]{figs/Current in circuit during damping.png}
    \caption{Graph of current in the circuit. Becomes zero after all the energy is dissipated}
    \label{fig:current_in_damping}
\end{figure}

\begin{figure}[H]
    \includegraphics[width=1\columnwidth]{figs/Power Dissipated across Resistor.png}
    \caption{Power Dissipated across resistor , $R=1\Omega$}
    \label{fig:power_in_damping}
\end{figure}

\end{document}

